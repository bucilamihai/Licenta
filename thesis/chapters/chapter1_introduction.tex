\chapter{Introducere}
\label{intro}
În contextul erei digitale, cantitatea de informație disponibilă în mediul online a crescut exponențial, generând nevoia de metode eficiente de filtrare și de selecție a conținutului.
Astfel, sistemele de recomandare au devenit un element indispensabil în numeroase arii, precum comerț electronic, platforme de streaming și rețele sociale.
Aceste sisteme oferă clienților sugestii personalizate, contribuind la o experiență de utilizare superioară și la retenția acestora.
Lucrarea de față surprinde potențialul unui astfel de sistem într-un context diferit, acela al interacțiunilor umane bazate pe interese similare.


\section{Obiective}
\label{section:ch1sec1}
Principalul obiectiv al lucrării este de a evidenția rolul și utilitatea sistemelor de recomandare într-o aplicație a cărui scop este asocierea între persoane pe baza intereselor comune.
Lucrarea își propune să exploreze și să compare principalele metode utilizate în recomandare: filtrarea bazată pe conținut și filtrarea colaborativă.
De asemenea, se vor analiza și principalele obstacole asociate acestor tehnici, cum ar fi problema sinonimiei, confidențialitatea și încrederea în sistem.
În plus, proiectul urmărește să demonstreze aplicabilitatea acestor concepte prin punerea lor în practică într-un sistem funcțional.

\section{Motivație}
\label{section:ch1sec2}
În zilele noastre interacțiunile umane devin din ce în ce mai mediate de tehnologie, astfel nevoia de a crea conexiuni relevante și veritabile între persoane devine tot mai evidentă.
Deși scopul inițial al rețelelor sociale era de a aduce mai aproape persoanele unele de altele, în prezent ele nu mai răspund la fel de eficient dorinței utilizatorilor de a întâlni persoane cu interese similare.
Această lucrare pornește de la o motivație personală: dorința de a utiliza tehnologia nu doar pentru informare și divertisment, ci și pentru a stimula relațiile sociale.
Se încearcă, cu ajutorul unui sistem de asociere bazat pe hobby-uri, promovarea unui tip de interacțiune pozitivă bazată pe existența unor pasiuni comune între indivizi.

\section{Structura lucrării}
\label{section:ch1sec3}

Această lucrare este structurată în 5 capitole care prezintă atât aspecte teoretice, cât și aspecte practice ale subiectului abordat. 
Capitolul 1 prezintă o introducere a temei propuse, cu atenția îndreptată spre obiectivele lucrării și motivația acesteia. 
În capitol 2 se oferă detalii despre câteva soluții existente din domeniul sistemelor de recomandare.
Capitolul 3 este dedicat bazei teoretice.
Soluția software se descrie în capitolul 4. 
În capitolul 5 sunt prezentate concluziile și perspectivele de viitor.

\section{Utilizarea instrumentelor de inteligență artificială generative}
\label{section:ch1sec4}
Pe parcursul elaborării lucrării, au fost folosite unele instrumente bazate pe IA generativă, precum ChatGPT, Claude și GitHub Copilot.
Aceste unelte s-au dovedit utile pentru realizarea unor căutări avansate legate de subiectele specifice și pentru reformularea ideilor în scopul coerenței și clarității.
În zona dezvoltării software, Copilot și ChatGPT au contribuit la identificarea și remedierea rapidă a erorilor, dar și la verificarea funcționalităților implementate prin reviziurea automată a modificărilor în cadrul sistemului de control al versiunilor.
Este important de menționat că aceste instrumente nu înlocuiesc aportul uman asupra procesului de cercetare și scriere academică.
Din acest motiv, responsabilitatea pentru validarea informațiilor, corectitudinea soluțiilor tehnice și asigurarea originalității aparține în totalitate autorilor lucrării.


