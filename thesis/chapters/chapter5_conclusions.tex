\chapter{Concluzii}
\label{conclusions}

Această lucrare a avut ca scop dezvoltarea unei aplicații software care favorizează interacțiunea dintre utilizatorii cu interese comune, prin folosirea unei metode de asociere a persoanelor bazată pe similaritatea hobby-urilor.
Am analizat diversi algoritmi utilizați în sistemele de recomandare existente, atât în domeniul comerțului online, cât și al rețelelor sociale, pentru a aprofunda modul în care pot fi aplicate cu scopul conectării între persoane.
\par
Aplicația realizată permite utilizatorilor să își aleagă hobby-urile și să descopere alți utilizatori cu interese similare.
Asocierea se bazează pe identificarea unui scor de similaritate, calculat în funcție de hobby-urile comune.
Pentru a îmbunătăți relevanța rezultatelor, algoritmul poate fi optimizat în mai multe direcții: integrarea unor date demografice, precum vârsta, genul și locația utilizatorilor, ponderarea hobby-urilor în funcție de frecvența lor în rândul persoanelor --- hobby-urile comune ca muzica sau desenul pot avea un impact redus asupra calculul similarității, în timp ce cele rare sau specifice pot fi considerate mai semnificative în definirea profilului unui utilizator, dar și luarea în calcul a categoriilor de hobby-uri --- pot exista compatibilități chiar dacă două persoane nu împărtășesc exact aceleași hobby-uri, dar au interese legate de activități de același tip.
\par
Privind aplicația, există câteva direcții de dezvoltare care includ îmbunătățirea interfeței grafice pentru oferirea unei experiențe intuitive utilizatorilor, precum și adăugarea unei funcționalități de comunicare, care să permită utilizatorilor cu interese similare să inițieze conversații.
\par
Așadar, lucrarea evidențiaza potențialul unui sistem de recomandare axat pe interese comune pentru facilitarea interacțiunilor sociale. 
Optimizările propuse la nivel de algoritm și îmbunătățirile suplimentare aduse aplicației pot transforma aplicația într-o platformă eficientă de conectare între persoane cu profiluri compatibile.