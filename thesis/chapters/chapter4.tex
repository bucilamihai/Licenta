\chapter{Aplicație software}
\label{chap:ch4}

\section{Arhitectura}
\label{sec:ch4sec1}

Aplicația este alcătuită din 3 componente: interfața utilizator (frontend), partea de server (backend) și sistemul de recomandare. 
Partea de frontend este dezvoltată în Ionic React, tehnologie ce îmbină librăria React, scrisă în limbajul de programare JavaScript, 
utilizată pentru a construi interfețe utilizator și Ionic, un instrument ce ajută la dezvoltarea aplicațiilor cross-platform. 
Folosind Ionic, aplicația poate fi instalată pe mai multe platforme --- web, mobile (Android, iOS), desktop --- utilizând același cod sursă. 
Această abordare eficientizează dezvoltarea și întreținerea aplicațiilor, permițând o experiență de utilizare uniformă, indiferent de dispozitiv folosit. 

\subsection{Frontend}
\label{subsec:ch4sec1sub1}
Interfața utilizator a aplicației este compusă din rute, pagini și componente. 
Paginile corespund unor ecrane complete din aplicație, iar componentele sunt părți reutilizabile ale aplicației, precum un formular sau buton personalizat.
Componentele primesc proprietăți și returnează elemente de interfață utilizator.
Rutele definesc ce pagină trebuie afișată atunci când utilizatorul accesează o anumită adresă (URL). 
De exemplu, ruta \texttt{/login} afișează pagina de autentificare, ruta \texttt{/register} afișează pagina de înregistrare ș.a.m.d.

\par
Frontend-ul comunică cu serverul prin apeluri HTTP, datele fiind transmise în formatul JSON.
În cadrul aplicației a fost creat fișierul \texttt{api.ts} care grupează toate cererile care sunt trimise către backend. 
Acest serviciu oferă funcții precum \texttt{register}, \texttt{login} și \texttt{saveHobbies}.
Aplicația folosește biblioteca Axios, o soluție populară și eficientă pentru manipularea cererilor HTTP.


\subsection{Backend}
\label{subsec:ch4sec1sub2}

\subsection{Sistemul de recomandare}
\label{subsec:ch4sec1sub3}

\section{Funcționalități}
\label{sec:ch4sec2}

\section{Testare}
\label{sec:ch4sec3}

\section{Manual de utilizare}
\label{sec:ch4sec4}