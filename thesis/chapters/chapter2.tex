\chapter{Soluții similare în domeniul sistemelor de recomandare}
\label{chap:ch2}

Sistemele de recomandare sunt folosite la scară largă în multe aplicații, oferind conținut personalizat și o experientă de utilizare îmbunătățită. 
Aplicațiile ce le folosesc variază, de la platforme de comerț online la rețele sociale.
Scopul acestora este de a ajuta utilizatorii să obțină conținut relevant din cantitatea mare de informație care există, simplificând procesul de căutare.
Astfel de soluții contribuie semnificativ creșterea veniturilor companiilor și la fidelizarea utilizatorilor.
Algoritmii folosiți în astfel de sisteme sunt filtrarea bazată pe conținut, filtrarea colaborativă sau soluții hibride, fiecare cu avantajele și limitările sale.
Pe lângă preferințele redate direct de utilizatori, sistemele moderne de recomandare țin cont și de context: locație, dispozitiv folosit sau momentul zilei.
Cu toate acestea, tehnicile se confruntă cu anumite provocări, precum problema utilizatorilor noi, lipsa de diversitate în recomandări sau lipsa de transparență a algoritmilor.

\section{Platforme de vânzări și streaming}
\label{sec:ch2sec1}
\subsection*{Amazon}
Amazon este una dintre cele mai cunoscute platforme de comerț care utilizează sisteme de recomandare.
Acesti algoritmi sunt integrați în întreaga aplicație, de la pagina principală până la paginile individuale ale produselor.
Platforma folosește tehnologii avansate de recomandare și de analiză a datelor pentru a afla comportamentul utilizatorilor.
Datele relevante pentru sistem sunt: produsele vizualizate, adăugate în coș sau cumpărate, recenzii, istoricul căutărilor, dar și comportamentul altor utilizatori cu profil similar. 
Toate aceste date sunt folosite pentru a crea profiluri detaliate de utilizator, cu rol crucial în oferirea de recomandări personalizate.
Pe baza acestora, Amazon implementează metode precum filtrarea bazată pe conținut (pentru a recomanda produse similare celor accesate în trecut) și filtrarea colaborativă (cu scopul de a sugera produse populare în rândul utilizatorilor cu interese comune).
Sistemul lor de recomandare este considerat un factor cheie în maximizarea eficienței vânzărilor, ajutând la retenția clienților și stimulând cumpărăturile recurente\cite{ahmed2022amazon, smith2017two}.

\subsection*{Netflix}
Netflix este un exemplu de serviciu de tip streaming care implementează algoritmi de recomandare pentru a îmbunătăți experiența utilizatorilor și pentru a crește nivelul de satisfacție al acestora.
Fără acest sistem, platforma nu s-ar bucura de succesul pe care îl are.
La fel ca Amazon, Netflix folosește o combinație de tehnici, adaptate pentru a oferi sugestii personalizate. 
Sistemul ia în considerare date ce privesc durata și istoricul vizionărilor, rating-ul acordat filmelor și interacțiunile anterioare.
Impactul acestor tehnologii este semnificativ, astfel că Netflix reușește să crească timpul petrecut în aplicație și să mărească gradul de fidelizare a utilizatorilor.
Mai mult, recomandările ajută abonații să descopere titluri noi, pe care probabil nu le-ar fi căutat, sporind aprecierea lor pentru platformă\cite{chiny2022netflix, gomez2015netflix}.

\section{Rețele sociale}
\label{sec:ch2sec2}
\subsection*{Facebook}
Facebook este una dintre cele mai utilizate rețele sociale la nivel mondial.
Sistemele de recomandare sunt esențiale în modul în care platforma afișează noutățile, sugerează prieteni sau grupează pagini relevante pentru fiecare persoană.
Algoritmii aplicației analizează numeroși factori precum interacțiunile anterioare (reacții, comentarii, distribuiri), timpul petrecut pe anumite tipuri de postări, relațiile dintre utilizatori, dar și preferințele implicite, deduse din comportament.
Aceste informații sunt prelucrate în timp real pentru a determina ce conținut este cel mai relevant. 
Se poate observa că acest mod de personalizare influențează modul în care utilizatorii percep realitatea digitală, întrucât conținutul afișat este filtrat și ordonat în funcție de relevanță, nu de cronologie sau obiectivitate\cite{baatarjav2008group}.
\par
Un alt aspect important este faptul că platforma nu oferă doar recomandări explicite (sugestii de prieteni sau grupuri), ci și implicite, prin modul în care organizează și filtrează fluxul de informații. 
Acestea optimizează și afișarea reclamelor, asigurând o potrivire mai bună între interesele utilizatorilor și ofertele companiilor, mărind astfel eficiența campaniilor publicitare și veniturile platformei.
Astfel, persoanele care utilizează aplicația sunt predispuse să petreacă mai mult timp explorând conținut, interacționând cu prietenii și postările acestora, ceea ce determină un nivel ridicat de implicare și angajament față de platformă.
Prin utilizarea acestor tehnologii, Facebook nu doar îmbunătățește experiența individuală, ci și crește implicarea utilizatorilor, contribuind la menținerea unei baze de date active\cite{heimbach2015value}.
\par
Cu toate acestea, folosirea excesivă a algoritmilor poate conduce la crearea unor „bule informaționale”\cite{nguyen2014exploring}, unde utilizatorii sunt expuși doar la un anumit tip de conținut, ce le confirmă convingerile existente, limitând diversitatea opiniilor și a informațiilor la care o persoană poate avea acces.
În tot acest timp, personalizarea exagerată influențează percepția asupra realității și poate reduce capacitatea de a descoperi perspective noi sau diferite.

\section{Abordări existente în conectarea utilizatorilor bazată pe interese}
\label{sec:ch2sec3}
O direcție importantă în cadrul sistemelor de recomandare este reprezentată de metode de asociere între utilizatori, în functie de interesele comune.
Aceste abordări sunt des întâlnite în aplicații de socializare, unde rolul lor este de a facilita conexiunile relevante între persoanele cu profiluri similare.
\par
În unul dintre articolele de specialitate\cite{tsakalakis2018improved}, autorii propun să abordeze problema recomandării de prietenii prin dezvoltarea unei metode avansate de calcul a similarității între utilizatori.
Obiectivul principal al algoritmului lor este de a conecta persoane cu interese comune care se află în aceeași zonă geografică.
Autorii au evaluat performanțele metodei proiectate pe un eșantion de 286 de utilizatori prin compararea lor cu alte metode clasice de calcul ale similarității existente în literatură, obținând rezultate superioare în contextul specific.