\chapter{Fundamente teoretice în sisteme de recomandare}
\label{chap:ch3}

\section{Filtrare bazată pe conținut}
\label{sec:ch3sec1}

Filtrarea bazată pe conținut este o tehnică de recomandare care analizează atributele unui element sau ale unei persoane pentru a genera sugestii.
În general, această metodă este utilizată cu scopul de a oferi recomandări personalizate în funcție de profilul utilizatorilor.
Sistemul identifică similarități între profiluri și sugerează elemente sau persoane care corespund cel mai bine preferințelor utilizatorului.
\cite{kumar2018recommendation}

\subsection{}

\section{Filtrare colaborativă}
\label{sec:ch3sec2}