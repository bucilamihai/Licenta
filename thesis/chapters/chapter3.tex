\chapter{Fundamente teoretice în sisteme de recomandare}
\label{chap:ch3}

\section{Filtrare bazată pe conținut}
\label{sec:ch3sec1}

Filtrarea bazată pe conținut este o tehnică de recomandare care analizează atributele unui element sau ale unei persoane pentru a genera sugestii.
În general, această metodă este utilizată cu scopul de a oferi recomandări personalizate în funcție de profilul utilizatorilor.
Sistemul identifică similarități între profiluri și sugerează elemente sau persoane care corespund cel mai bine preferințelor utilizatorului.
\cite{kumar2018recommendation}
\par
Această metodă este des utilizată în platforme precum Spotify, YouTube sau Netflix, în special pentru a recomanda conținut similar cu cel deja consumat. 
De exemplu, un utilizator care ascultă muzică jazz va primi mai frecvent recomandări din aceeași zonă muzicală, fără a fi nevoie de comparații cu preferințele altor utilizatori. 
Avantajul major al acestui tip de filtrare constă în caracterul său personalizat, axat strict pe gusturile fiecărui utilizator, fără a fi influențat de tendințele generale ale comunității.
\par
Cu toate acestea, filtrarea bazată pe conținut are și unele limitări. 
Una dintre cele mai notabile este lipsa diversității — sistemul poate rămâne ”prizonier” în tiparele stabilite inițial și poate recomanda doar conținut foarte similar cu ceea ce a fost deja consumat. 
În plus, eficiența sistemului depinde de calitatea datelor și de granularitatea atributelor folosite pentru descrierea obiectelor.

\subsection{Principiu de funcționare}
Metoda se bazează pe construirea unui profil pentru fiecare utilizator, care reflectă preferințele sale în funcție de conținutul anterior accesat sau evaluat. 
Acest profil este comparat cu descrierile altor elemente disponibile, iar sistemul recomandă acele elemente care prezintă cele mai mari similarități.
De regulă, se folosesc tehnici de procesare a limbajului natural, metode probabilistice sau modele de învățare automată.
Una dintre particularitățile filtrării bazate pe conținut este faptul că nu are nevoie de date despre alți utilizatori pentru a funcționa. Recomandările sunt generate exclusiv pe baza preferințelor și comportamentului.
Acest lucru oferă o flexibilitate ridicată, în special atunci când predilecțiile se schimbă. Pe de altă parte, această abordare presupune existența unor descrieri detaliate ale elementelor din sistem, ceea ce poate deveni o provocare atunci când aceste informații lipsesc sau sunt dificil de structurat.
\cite{ISINKAYE2015261}
\section{Filtrare colaborativă}
\label{sec:ch3sec2}
\section{Provocări și limitări ale metodelor de recomandare}
\label{sec:ch3sec3}
Deși sistemele de recomandare bazate pe conținut și colaborative au un rol esențial în personalizarea experienței utilizatorilor, acestea se confruntă cu o serie de limitări care pot afecta eficiența și precizia recomandărilor. 
Printre cele mai întâlnite provocări se numără lipsa datelor suficiente pentru utilizatorii noi, dificultatea de a surprinde preferințele în schimbare și complexitatea modelării relațiilor între elemente sau utilizatori. 
În continuare sunt prezentate câteva dintre aceste probleme, împreună cu implicațiile lor asupra performanței sistemelor de recomandare.

\subsection{Problema ”Cold start”}
\label{subsec:ch3sec3sub1}
Problema ”Cold start” reprezintă o situație care des întâlnită în domeniul sistemelor de recomandare. 
Aceasta apare atunci când nu există suficiente date despre un utilizator sau un obiect pentru a face recomandări precise.
În cazul unui utilizator nou, sistemul nu dispune de informații suficiente despre acesta, făcând dificilă furnizarea sugestiilor.
În mod similar, în cazul unui obiect nou, 
Astfel, se poate ajunge la recomandări inexacte ce afectează negativ experiența utilizatorului.
Soluțiile pentru această problemă includ utilizarea tehnicilor de recomandare bazate pe conținut, care nu depind de interacțiuni anterioare, 
în detrimentul recomandărilor colaborative sau combinarea mai multor metode pentru o acuratețe îmbunătățită în etapele incipiente de folosire.
Concret, sistemul ar trebui fie să ofere posibilitatea unui utilizator nou să evalueze anumite articole sau să întrebe explicit despre gusturile acestuia pentru a-i construi un profil, 
fie să dea recomandări preliminare bazate pe informații demografice sau alte date disponibile.
\cite{kumar2018recommendation}


\subsection{Sinonimie}
\label{subsec:ch3sec3sub2}

\subsection{Confidențialitate și securitate}
\label{subsec:ch3sec3sub3}