\documentclass[12pt]{report} 

% language may be romanian or english (default is english)
% type may be bachelor or master (default is bachelor)
\usepackage[language=romanian, type=bachelor]{style}

%\geometry{a4paper,top=2.5cm,left=3cm,right=2.5cm,bottom=2.5cm}
%in style
%controlling the appearance of your headers and footers
\usepackage{fancyhdr}    
\pagestyle{fancy}
\lhead{}
\chead{}
\renewcommand{\headrulewidth}{0.2pt}
\renewcommand{\footrulewidth}{0.2pt}

\usepackage{listings}
\usepackage{xcolor}
\renewcommand{\lstlistingname}{Secvența de cod}
\lstdefinestyle{pythonstyle}{
    language=Python,
    basicstyle=\ttfamily\footnotesize,
    keywordstyle=\color{blue}\bfseries,
    commentstyle=\color{green!60!black}\itshape,
    stringstyle=\color{red},
    numberstyle=\tiny\color{gray},
    numbers=left,
    numbersep=8pt,
    stepnumber=1,
    backgroundcolor=\color{gray!5},
    frame=single,
    frameround=tttt,
    rulecolor=\color{gray!30},
    breaklines=true,
    breakatwhitespace=false,
    tabsize=4,
    showspaces=false,
    showstringspaces=false,
    showtabs=false,
    captionpos=b,
    aboveskip=\medskipamount,
    belowskip=\medskipamount
}
\lstset{style=pythonstyle}

\begin{document}

\specialization{INFORMATICĂ ÎN LIMBA ROMÂNĂ}	
\title{Conectarea persoanelor bazată pe interese comune folosind sisteme de recomandare}					   
\author{Bucilă Mihai-Cristian}											
\supervisor{Asist. Drd. Coste Claudia-Ioana \\ Prof. univ. Dr. Andreica Anca-Mirela}				
				
\maketitle


\newpage
\thispagestyle{empty}
\mbox{}
\newpage
\pagenumbering{roman} 

\cleardoublepage
ABSTRACT
\vspace{0.5cm}	
\hrule
\vspace{0.5cm}	
%\cleardoublepage

\par
The abundance of information on the internet has driven the rapid evolution of recommender systems.
Their ability to provide personalized suggestions has led to widespread adoption in high-demand areas like e-commerce and social networking platforms, enhancing both user experience and retention.
\par
This thesis presents the core techniques underlying recommendation systems --- content-based filtering and collaborative filtering --- and explores the challenges that can arise when implementing these methods, such as the cold start problem, synonymy and privacy concerns.
The main objective of this work is to demonstrate the utility of a recommender system in the context of an application that matches users based on their shared hobbies.
The application analyzes user-provided hobby data and uses a simplified matching algorithm to identify potential matches within a shared interest space.
\par
This work is motivated by the idea that recommender systems can have an impact beyond commercial platforms and enhance social connectivity and community building.
By combining the theoretical knowledge with a concrete implementation, the project shows how recommendation techniques can be adapted to support social engagement and personal development.
The application serves not only as a case study, but also as a proof of concept for using technology to bring people together in a more intentional way.
\par
This paper is organized into five chapters that address both theoretical background and practical solution.
The first chapter introduces the topic, exposing the main objectives and the motivation behind the research.
The second chapter reviews several existing recommendatation system solutions and studies.
Chapter 3 provides the theoretical concepts which are essential to understand the proposed approach.
Chapter 4 presents the developed software solution.
Finally, the last chapter summarizes the work and mentions potential directions for future work.


\tableofcontents


\newpage
\pagenumbering{arabic}

\chapter{Introducere}
\label{intro}
În contextul erei digitale, cantitatea de informație disponibilă în mediul online a crescut exponențial, generând nevoia de metode eficiente de filtrare și de selecție a conținutului.
Astfel, sistemele de recomandare au devenit un element indispensabil în numeroase arii, precum comerț electronic, platforme de streaming și rețele sociale.
Aceste sisteme oferă clienților sugestii personalizate, contribuind la o experiență de utilizare superioară și la retenția acestora.
Lucrarea de față surprinde potențialul unui astfel de sistem într-un context diferit, acela al interacțiunilor umane bazate pe interese similare.


\section{Obiective}
\label{section:ch1sec1}
Principalul obiectiv al lucrării este de a evidenția rolul și utilitatea sistemelor de recomandare într-o aplicație a cărui scop este asocierea între persoane pe baza intereselor comune.
Lucrarea își propune să exploreze și să compare principalele metode utilizate în recomandare: filtrarea bazată pe conținut și filtrarea colaborativă.
De asemenea, se vor analiza și principalele obstacole asociate acestor tehnici, cum ar fi problema sinonimiei, confidențialitatea și încrederea în sistem.
În plus, proiectul urmărește să demonstreze aplicabilitatea acestor concepte prin punerea lor în practică într-un sistem funcțional.

\section{Motivație}
\label{section:ch1sec2}
În zilele noastre interacțiunile umane devin din ce în ce mai mediate de tehnologie, astfel nevoia de a crea conexiuni relevante și veritabile între persoane devine tot mai evidentă.
Deși scopul inițial al rețelelor sociale era de a aduce mai aproape persoanele unele de altele, în prezent ele nu mai răspund la fel de eficient dorinței utilizatorilor de a întâlni persoane cu interese similare.
Mai mult, studii de la Universitatea din Pennsylvania \cite{hunt2018no} au stabilit că utilizarea îndelungată a rețelelor sociale duce la stări de singurătate și anxietate.
Experimentul în care au fost implicați 143 de studenți, împărțiți în două grupuri (unul care și-a limitat utilizarea platformelor sociale la 30 de minute, timp de 3 săptămâni și altul de control) a arătat efectele nocive pe care folosirea necontrolată le are asupra indivizilor.
În urma acestuia, au fost observate scăderi semnificative ale stărilor de depresie și anxietate, alături de o îmbunătățire a stimei de sine.
Persoanele care au participat la acest studiu au remarcat cât de importantă este folosirea conștientă a acestor platforme.
\par
Motivația personală din spatele acestei lucrări este dorința de a utiliza tehnologia nu doar pentru informare și divertisment, ci și pentru a stimula relațiile sociale.
O cercetare din domeniu \cite{xiao2018common} confirmă ideea că tinerii formează relații de prietenie mai puternice cu persoane care au interese apropiate.
Autorii arată cum sportul, muzica și activitățile extracurriculare facilitează interacțiunile sociale.
Toate aceste dovezi susțin ideea unui sistem de asociere bazat pe hobby-uri pentru promovarea unui tip de interacțiune pozitivă bazată pe existența unor pasiuni comune între indivizi.

\section{Structura lucrării}
\label{section:ch1sec3}

Această lucrare este structurată în 5 capitole care prezintă atât aspecte teoretice, cât și aspecte practice ale subiectului abordat. 
Capitolul 1 prezintă o introducere a temei propuse, cu atenția îndreptată spre obiectivele lucrării și motivația acesteia, menționând studii și cercetări pentru susținerea acesteia.
În capitol 2 se oferă detalii despre câteva soluții existente din domeniul sistemelor de recomandare, realizând o analiză comparativă a acestor abordări.
Capitolul 3 este dedicat principiilor fundamentale ale sistemelor de recomandare, oferind un punct de plecare pentru înțelegerea și implementarea soluției propuse.
Soluția software se descrie în capitolul 4, care prezintă arhitectura sistemului, principalele funcționalități și un scurt ghid de utilizare.
În capitolul 5 sunt prezentate concluziile și perspectivele de viitor.

\section{Utilizarea instrumentelor de inteligență artificială generative}
\label{section:ch1sec4}
Pe parcursul elaborării lucrării, au fost folosite unele instrumente bazate pe IA generativă, precum ChatGPT, Claude și GitHub Copilot.
Aceste unelte s-au dovedit utile pentru realizarea unor căutări avansate legate de subiectele specifice și pentru reformularea ideilor în scopul coerenței și clarității.
În zona dezvoltării software, Copilot și ChatGPT au contribuit la identificarea și remedierea rapidă a erorilor, dar și la verificarea funcționalităților implementate prin reviziurea automată a modificărilor în cadrul sistemului de control al versiunilor.
Este important de menționat că aceste instrumente nu înlocuiesc aportul uman asupra procesului de cercetare și scriere academică.
Din acest motiv, responsabilitatea pentru validarea informațiilor, corectitudinea soluțiilor tehnice și asigurarea originalității aparține în totalitate autorilor lucrării.



%\addcontentsline{toc}{chapter}{Introducere}
%\addcontentsline{toc}{chapter}{Introduction}

\chapter{Soluții similare în domeniul sistemelor de recomandare}
\label{chap:ch2}

Sistemele de recomandare sunt folosite la scară largă în multe aplicații, oferind conținut personalizat și o experientă de utilizare îmbunătățită. 
Aplicațiile ce le folosesc variază, de la platforme de comerț online la rețele sociale.
Scopul acestora este de a ajuta utilizatorii să obțină conținut relevant din cantitatea mare de informație care există, simplificând procesul de căutare.
Astfel de soluții contribuie semnificativ creșterea veniturilor companiilor și la fidelizarea utilizatorilor.
Algoritmii folosiți în astfel de sisteme sunt filtrarea bazată pe conținut, filtrarea colaborativă sau soluții hibride, fiecare cu avantajele și limitările sale.
Pe lângă preferințele redate direct de utilizatori, sistemele moderne de recomandare țin cont și de context: locație, dispozitiv folosit sau momentul zilei.
Cu toate acestea, tehnicile se confruntă cu anumite provocări, precum problema utilizatorilor noi, lipsa de diversitate în recomandări sau lipsa de transparență a algoritmilor.

\section{Platforme de vânzări și streaming}
\label{sec:ch2sec1}
\subsection*{Amazon}
Amazon este una dintre cele mai cunoscute platforme de comerț care utilizează sisteme de recomandare.
Acesti algoritmi sunt integrați în întreaga aplicație, de la pagina principală până la paginile individuale ale produselor.
Platforma folosește tehnologii avansate de recomandare și de analiză a datelor pentru a afla comportamentul utilizatorilor.
Datele relevante pentru sistem sunt: produsele vizualizate, adăugate în coș sau cumpărate, recenzii, istoricul căutărilor, dar și comportamentul altor utilizatori cu profil similar. 
Toate aceste date sunt folosite pentru a crea profiluri detaliate de utilizator, cu rol crucial în oferirea de recomandări personalizate.
Pe baza acestora, Amazon implementatează metode precum filtrarea bazată pe conținut (pentru a recomanda produse similare celor accesate în trecut) și filtrarea colaborativă (cu scopul de a sugera produse populare în rândul utilizatorilor cu interese comune).
Sistemul lor de recomandare este considerat un factor cheie în maximizarea eficienței vânzărilor, ajutând la retenția clienților și stimulând cumpărăturile recurente\cite{ahmed2022amazon, smith2017two}.

\subsection*{Netflix}
Netflix este un exemplu de serviciu de tip streaming care implementează algoritmi de recomandare pentru a îmbunătăți experiența utilizatorilor și pentru a crește nivelul de satisfacție al acestora.
Fără acest sistem, platforma nu s-ar bucura de succesul pe care îl are.
La fel ca Amazon, Netflix folosește o combinație de tehnici, adaptate pentru a oferi sugestii personalizate. 
Sistemul ia în considerare date ce privesc durata și istoricul vizionărilor, rating-ul acordat filmelor și interacțiunile anterioare.
Impactul acestor tehnologii este semnificativ, astfel că Netflix reușește să crească timpul petrecut în aplicație și să mărească gradul de fidelizare a utilizatorilor.
Mai mult, recomandările ajută abonații să descopere titluri noi, pe care probabil nu le-ar fi căutat, sporind aprecierea lor pentru platformă\cite{chiny2022netflix, gomez2015netflix}.

\section{Rețele sociale}
\label{sec:ch2sec2}
\subsection*{Facebook}
Facebook este una dintre cele mai utilizate rețele sociale la nivel mondial.
Sistemele de recomandare sunt esențiale în modul în care platforma afișează noutățile, sugerează prieteni sau grupează pagini relevante pentru fiecare persoană.
Algoritmii aplicației analizează numeroși factori precum interacțiunile anterioare (reacții, comentarii, distribuiri), timpul petrecut pe anumite tipuri de postări, relațiile dintre utilizatori, dar și preferințele implicite, deduse din comportament.
Aceste informații sunt prelucrate în timp real pentru a determina ce conținut este cel mai relevant. 
Se poate observa că acest mod de personalizare influențează modul în care utilizatorii percep realitatea digitală, întrucât conținutul afișat este filtrat și ordonat în funcție de relevanță, nu de cronologie sau obiectivitate\cite{baatarjav2008group}.
\par
Un alt aspect important este faptul că platforma nu oferă doar recomandări explicite (sugestii de prieteni sau grupuri), ci și implicite, prin modul în care organizează și filtrează fluxul de informații. 
Acestea optimizează și afișarea reclamelor, asigurând o potrivire mai bună între interesele utilizatorilor și ofertele companiilor, mărind astfel eficiența campaniilor publicitare și veniturile platformei.
Astfel, persoanele care utilizează aplicația sunt predispuse să petreacă mai mult timp explorând conținut, interacționând cu prietenii și postările acestora, ceea ce determină un nivel ridicat de implicare și angajament față de platformă.
Prin utilizarea acestor tehnologii, Facebook nu doar îmbunătățește experiența individuală, ci și crește implicarea utilizatorilor, contribuind la menținerea unei baze de date active\cite{heimbach2015value}.
\par
Cu toate acestea, folosirea excesivă a algoritmilor poate conduce la crearea unor „bule informaționale”\cite{nguyen2014exploring}, unde utilizatorii sunt expuși doar la un anumit tip de conținut, ce le confirmă convingerile existente, limitând diversitatea opiniilor și a informațiilor la care o persoană poate avea acces.
În tot acest timp, personalizarea exagerată influențează percepția asupra realității și poate reduce capacitatea de a descoperi perspective noi sau diferite.

\section{Abordări existente în conectarea utilizatorilor bazată pe interese}
\label{sec:ch2sec3}
O direcție importantă în cadrul sistemelor de recomandare este reprezentată de metode de asociere între utilizatori, în functie de interesele comune.
Aceste abordări sunt des întâlnite în aplicații de socializare, unde rolul lor este de a facilita conexiunile relevante între persoanele cu profiluri similare.
\par
În unul dintre articolele de specialitate\cite{tsakalakis2018improved}, autorii propun să abordeze problema recomandării de prietenii prin dezvoltarea unei metode avansate de calcul a similarității între utilizatori.
Obiectivul principal al algoritmului lor este de a conecta persoane cu interese comune care se află în aceeași zonă geografică.
Autorii au evaluat performanțele metodei proiectate pe un eșantion de 286 de utilizatori prin compararea lor cu alte metode clasice de calcul ale similarității existente în literatură, obținând rezultate superioare în contextul specific.
\chapter{Fundamente teoretice în sisteme de recomandare}
\label{chap:ch3}

\section{Filtrare bazată pe conținut}
\label{sec:ch3sec1}

Filtrarea bazată pe conținut este o tehnică de recomandare care analizează atributele unui element sau ale unei persoane pentru a genera sugestii.
În general, această metodă este utilizată cu scopul de a oferi recomandări personalizate în funcție de profilul utilizatorilor.
Sistemul identifică similarități între profiluri și sugerează elemente sau persoane care corespund cel mai bine preferințelor utilizatorului.
\cite{kumar2018recommendation}
\par
Această metodă este des utilizată în platforme precum Spotify, YouTube sau Netflix, în special pentru a recomanda conținut similar cu cel deja consumat. 
De exemplu, un utilizator care ascultă muzică jazz va primi mai frecvent recomandări din aceeași zonă muzicală, fără a fi nevoie de comparații cu preferințele altor utilizatori. 
Avantajul major al acestui tip de filtrare constă în caracterul său personalizat, axat strict pe gusturile fiecărui utilizator, fără a fi influențat de tendințele generale ale comunității.
\par
Cu toate acestea, filtrarea bazată pe conținut are și unele limitări. 
Una dintre cele mai notabile este lipsa diversității — sistemul poate rămâne ”prizonier” în tiparele stabilite inițial și poate recomanda doar conținut foarte similar cu ceea ce a fost deja consumat. 
În plus, eficiența sistemului depinde de calitatea datelor și de granularitatea atributelor folosite pentru descrierea obiectelor.

\subsection{Principiu de funcționare}
Metoda se bazează pe construirea unui profil pentru fiecare utilizator, care reflectă preferințele sale în funcție de conținutul anterior accesat sau evaluat. 
Acest profil este comparat cu descrierile altor elemente disponibile, iar sistemul recomandă acele elemente care prezintă cele mai mari similarități.
De regulă, se folosesc tehnici de procesare a limbajului natural, metode probabilistice sau modele de învățare automată.
Una dintre particularitățile filtrării bazate pe conținut este faptul că nu are nevoie de date despre alți utilizatori pentru a funcționa. Recomandările sunt generate exclusiv pe baza preferințelor și comportamentului.
Acest lucru oferă o flexibilitate ridicată, în special atunci când predilecțiile se schimbă. Pe de altă parte, această abordare presupune existența unor descrieri detaliate ale elementelor din sistem, ceea ce poate deveni o provocare atunci când aceste informații lipsesc sau sunt dificil de structurat.
\cite{ISINKAYE2015261}
\section{Filtrare colaborativă}
\label{sec:ch3sec2}
\section{Provocări și limitări ale metodelor de recomandare}
\label{sec:ch3sec3}
Deși sistemele de recomandare bazate pe conținut și colaborative au un rol esențial în personalizarea experienței utilizatorilor, acestea se confruntă cu o serie de limitări care pot afecta eficiența și precizia recomandărilor. 
Printre cele mai întâlnite provocări se numără lipsa datelor suficiente pentru utilizatorii noi, dificultatea de a surprinde preferințele în schimbare și complexitatea modelării relațiilor între elemente sau utilizatori. 
În continuare sunt prezentate câteva dintre aceste probleme, împreună cu implicațiile lor asupra performanței sistemelor de recomandare.

\subsection{Problema ”Cold start”}
\label{subsec:ch3sec3sub1}
Problema ”Cold start” reprezintă o situație care des întâlnită în domeniul sistemelor de recomandare. 
Aceasta apare atunci când nu există suficiente date despre un utilizator sau un obiect pentru a face recomandări precise.
În cazul unui utilizator nou, sistemul nu dispune de informații suficiente despre acesta, făcând dificilă furnizarea sugestiilor.
În mod similar, în cazul unui obiect nou, 
Astfel, se poate ajunge la recomandări inexacte ce afectează negativ experiența utilizatorului.
Soluțiile pentru această problemă includ utilizarea tehnicilor de recomandare bazate pe conținut, care nu depind de interacțiuni anterioare, 
în detrimentul recomandărilor colaborative sau combinarea mai multor metode pentru o acuratețe îmbunătățită în etapele incipiente de folosire.
Concret, sistemul ar trebui fie să ofere posibilitatea unui utilizator nou să evalueze anumite articole sau să întrebe explicit despre gusturile acestuia pentru a-i construi un profil, 
fie să dea recomandări preliminare bazate pe informații demografice sau alte date disponibile.
\cite{kumar2018recommendation}


\subsection{Sinonimie}
\label{subsec:ch3sec3sub2}

\subsection{Confidențialitate și securitate}
\label{subsec:ch3sec3sub3}
\chapter{Aplicație software}
\label{chap:ch4}

\section{Arhitectura}
\label{sec:ch4sec1}

Aplicația este alcătuită din 3 componente: interfața utilizator (frontend), partea de server (backend) și sistemul de recomandare. 
Partea de frontend este dezvoltată în Ionic React, tehnologie ce îmbină librăria React, scrisă în limbajul de programare JavaScript, 
utilizată pentru a construi interfețe utilizator și Ionic, un instrument ce ajută la dezvoltarea aplicațiilor cross-platform. 
Folosind Ionic, aplicația poate fi instalată pe mai multe platforme --- web, mobile (Android, iOS), desktop --- utilizând același cod sursă. 
Această abordare eficientizează dezvoltarea și întreținerea aplicațiilor, permițând o experiență de utilizare uniformă, indiferent de dispozitiv folosit. 

\subsection{Frontend}
\label{subsec:ch4sec1sub1}
Interfața utilizator a aplicației este compusă din rute, pagini și componente. 
Paginile corespund unor ecrane complete din aplicație, iar componentele sunt părți reutilizabile ale aplicației, precum un formular sau buton personalizat.
Componentele primesc proprietăți și returnează elemente de interfață utilizator.
Rutele definesc ce pagină trebuie afișată atunci când utilizatorul accesează o anumită adresă (URL). 
De exemplu, ruta \texttt{/login} afișează pagina de autentificare, ruta \texttt{/register} afișează pagina de înregistrare ș.a.m.d.

\par
Frontend-ul comunică cu serverul prin apeluri HTTP, datele fiind transmise în formatul JSON.
În cadrul aplicației a fost creat fișierul \texttt{api.ts} care grupează toate cererile care sunt trimise către backend. 
Acest serviciu oferă funcții precum \texttt{register}, \texttt{login} și \texttt{saveHobbies}.
Aplicația folosește biblioteca Axios, o soluție populară și eficientă pentru manipularea cererilor HTTP.


\subsection{Backend}
\label{subsec:ch4sec1sub2}

\subsection{Sistemul de recomandare}
\label{subsec:ch4sec1sub3}

\section{Funcționalități}
\label{sec:ch4sec2}

\section{Testare}
\label{sec:ch4sec3}

\section{Manual de utilizare}
\label{sec:ch4sec4}

\chapter{Concluzii}
\label{conclusions}

\par Concluzii ...

%\addcontentsline{toc}{chapter}{Concluzii}
%\addcontentsline{toc}{chapter}{Conclusions}

\bibliography{references}

\end{document}
